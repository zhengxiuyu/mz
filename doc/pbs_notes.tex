\documentclass[8pt]{article}
\usepackage{amsmath}

\title{\sf Some Notes on Position Based Fluids}
\author{\small Arno Luebke (arno.luebke@googlemail.com)}
\date{}

\begin{document}
\maketitle
\underline{Disclaimer}: This document is probably full of mistakes. Do not blindly trust any statement here.

\section*{\sf Some notes on kernels}
\begin{itemize}
  \item Kernels approximate the delta function
  \item $W(\vec{p}_i - \vec{p}_j, h)$ can be written more briefly as $W^h_{ij}$
  \item Kernels are symmetric, i.e. $W^h_{ij} = W^h_{ij}$
\end{itemize}

\section*{\sf The constraint gradient}
The constraint gradient can be derived as follows:
\begin{eqnarray}
\nabla_{\vec{p}_k}C_i = \nabla_{\vec{p}_k}\left(\frac{\rho_i}{\rho_0} - 1\right)=\frac{1}{\rho_0} \nabla_{\vec{p}_k}\rho_i =\frac{1}{\rho_0} \nabla_{\vec{p}_k}\sum_jW^h_{ij}=\frac{1}{\rho_0} \sum_j\nabla_{\vec{p}_k}W^h_{ij}.
\end{eqnarray}
Expanding the sum gives
\begin{eqnarray}
\nabla_{\vec{p}_k}C_i = \frac{1}{\rho_0}\left(\nabla_{\vec{p}_k}W^h_{i1} + \nabla_{\vec{p}_k}W^h_{i2} + ... + \nabla_{\vec{p}_k}W^h_{ik} + ... + \nabla_{\vec{p}_k}W^h_{iN}\right)
\end{eqnarray}
We note that if $k\neq i$, all terms except for $\nabla_{\vec{p}_k}W^h_{ik}$ vanish as they do not depend on $\vec{p}_k$. However, if $k = i$, all terms depend of $\vec{p}_k$. Hence,
\begin{eqnarray}
\nabla_{\vec{p}_k}C_i =\frac{1}{\rho}_0 \left\{\begin{array}{ll}
\sum_j \nabla_{\vec{p}_k}W^h_{ij}& {\rm if \ } k = i \\[0.25cm]
\nabla_{\vec{p}_k}W^h_{ik} & {\rm otherwise}
\end{array}\right. .
\end{eqnarray}
Note: This equation differs from Equation 8 in the original paper in that a minus sign for the second condition is missing. In my opinion this makes more sense when deriving the total position update.



\section*{\sf The total position update for particle $i$ : $\Delta \vec{p}_i$}
Applying the Newton step for each constraint function leads to:
\begin{eqnarray}
  \Delta \vec{p}_i = \lambda_1\nabla_{\vec{p}_i}C_1 + \lambda_2\nabla_{\vec{p}_i}C_2 + ... +\lambda_i\nabla_{\vec{p}_i}C_i + ... +\lambda_N\nabla_{\vec{p}_i}C_N.
\end{eqnarray}
Given the definition of $\nabla_{\vec{p}_k}C_i$ this gives
\begin{eqnarray*}
  \Delta \vec{p}_i &=& \frac{1}{\rho_0}\left(\lambda_1\nabla_{\vec{p}_i}W^h_{1i} + \lambda_2\nabla_{\vec{p}_i}W^h_{2i} + ... + \lambda_i\sum_{k\neq i}\nabla_{\vec{p}_i}W^h_{ik} + ... + \lambda_N\nabla_{\vec{p}_i}W^h_{Ni}\right)\\
  &=& \frac{1}{\rho_0}{\Big(}\lambda_1\nabla_{\vec{p}_i}W^h_{1i} + \lambda_2\nabla_{\vec{p}_i}W^h_{2i} + ... + (\lambda_i\nabla_{\vec{p}_i}W^h_{i1} + \lambda_i\nabla_{\vec{p}_i}W^h_{i2} +  ... \\
  && \quad \quad \quad + \lambda_i\nabla_{\vec{p}_i}W^h_{iN} )... + \lambda_N\nabla_{\vec{p}_i}W^h_{Ni}{\Big).}
\end{eqnarray*}
Due to kernel symmetry terms pair up as follows
\begin{eqnarray*}
   \Delta \vec{p}_i &=& \frac{1}{\rho_0}\Big((\lambda_i + \lambda_1)\nabla_{\vec{p}_i}W^h_{i1} + (\lambda_i + \lambda_2)\nabla_{\vec{p}_i}W^h_{i2} + ... + (\lambda_i + \lambda_N)\nabla_{\vec{p}_i}W^h_{iN}\Big)\\
           &=& \sum_j(\lambda_i + \lambda_j)\nabla_{\vec{p}_i}W^h_{ij}
\end{eqnarray*}
The sum above is over all particles. However, since $\nabla_{\vec{p}_i}W^h_{ij} = \vec{0}$ if particle $i$ and $j$ are not neighbors, implementations just sum over the neighbors of particle $i$. 


\section*{\sf{Kernels}}
\begin{itemize}
    \item In two dimensions: given an unnormalized kernel $W'(r, h)$. $W'(r, h)$ is normalized by $2\pi\int_{r=0}^h r W'(r, h) dr$.
    \item \emph{poly6} Kernel (2D)
        \begin{eqnarray}
        W_{poly6\_2D}(r, h) &=& \frac{4}{\pi h^8}\left\{\begin{array}{ll} (h^2 - r^2)^3 & {\rm if\ } r < h \\ 0 & {\rm otherwise}\end{array}\right.
        \end{eqnarray}
    \item \emph{spiky} Kernel (2D)
        \begin{eqnarray}
            W(\vec{x}, h) &=& \frac{10}{\pi h^5}(h - \Vert\vec{x}\Vert)^3 \\
            \nabla W(\vec{x}, h) &=& -\frac{30}{\pi h^5}(h - \Vert\vec{x}\Vert)^2\frac{\vec{x}}{\Vert\vec{x}\Vert}
        \end{eqnarray}
\end{itemize}


\end{document}
