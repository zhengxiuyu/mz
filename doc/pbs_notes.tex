\documentclass[10pt]{article}
\usepackage{amsmath}

\title{Some Notes on Position Based Fluids}
\author{Arno Luebke (arno.luebke@googlemail.com)}
\date{}

\begin{document}
\maketitle
\underline{General}
\begin{itemize}
    \item the vector $\bf p$ is a concatenation of particle positions, i.e. 
        \begin{eqnarray*}
            {\bf p} = ({\bf p}_1, {\bf p}_2, ..., {\bf p}_N) = (x_1, y_1, z_1,  x_2, y_2, z_2,  ..., x_N, y_N, z_N)
        \end{eqnarray*}
    
    \item similarly,
        \begin{eqnarray*}
            {\Delta \bf p} &=& ({\Delta\bf p}_1, {\Delta\bf p}_2, ..., {\Delta\bf p}_N) \\
            &=& (\Delta x_1, \Delta y_1, \Delta z_1, \Delta x_2, \Delta y_2, \Delta z_2, ..., \Delta x_N, \Delta y_N, \Delta z_N)
        \end{eqnarray*}
        
    \item Equation 4 expands to:
        \begin{eqnarray*}
            \begin{pmatrix}
                \Delta{\bf p}_1 \\
                \Delta{\bf p}_2 \\
                \vdots \\
                \Delta{\bf p}_N
            \end{pmatrix}
            &\approx&
            \left(\nabla C_1({\bf p}); \nabla C_2({\bf p}); \hdots; \nabla C_N({\bf p})\right)
            \begin{pmatrix}
                \lambda_1 \\
                \lambda_2 \\
                \vdots \\
                \lambda_N
            \end{pmatrix}\\
            &\approx&
            \begin{pmatrix}
                \nabla_{{\bf p}_1} C_1({\bf p}) &  \nabla_{{\bf p}_1} C_2({\bf p}) & \hdots  & \nabla_{{\bf p}_1} C_N({\bf p}) \\
                \nabla_{{\bf p}_2} C_1({\bf p}) &  \nabla_{{\bf p}_2} C_2({\bf p}) & \hdots  & \nabla_{{\bf p}_2} C_N({\bf p}) \\
                \vdots & \vdots & & \vdots \\
                \nabla_{{\bf p}_N} C_1({\bf p})  &  \nabla_{{\bf p}_N} C_2({\bf p}) & \hdots  & \nabla_{{\bf p}_N} C_N({\bf p}) \\
            \end{pmatrix}
            \begin{pmatrix}
                \lambda_1 \\
                \lambda_2 \\
                \vdots \\
                \lambda_N
            \end{pmatrix}\\            
        \end{eqnarray*}
        where $\nabla_{{\bf p}_k} C_i({\bf p}) = \partial C_i ({\bf p}) / \partial {\bf p}_k =  (\partial C_i({\bf p}) / \partial x_k, \partial C_i({\bf p}) / \partial y_k, \partial C_i({\bf p}) / \partial z_k)$
    
    \item Equation 6 expands to
        \begin{eqnarray*}
            \begin{pmatrix}
                C_1({\bf p}) \\
                C_2({\bf p}) \\
                \vdots \\
                C_N({\bf p})
            \end{pmatrix}
            +
            \begin{pmatrix}
                \nabla_{{\bf p}_1} C_1({\bf p}) &  \nabla_{{\bf p}_2} C_1({\bf p}) & \hdots  & \nabla_{{\bf p}_N} C_1({\bf p}) \\
                \nabla_{{\bf p}_1} C_2({\bf p}) &  \nabla_{{\bf p}_2} C_2({\bf p}) & \hdots  & \nabla_{{\bf p}_N} C_2({\bf p}) \\
                \vdots & \vdots & & \vdots \\
                \nabla_{{\bf p}_1} C_N({\bf p})  &  \nabla_{{\bf p}_2} C_N({\bf p}) & \hdots  & \nabla_{{\bf p}_N} C_N({\bf p}) \\
            \end{pmatrix} \cdot \\
            {\vspace{6cm}}
            \begin{pmatrix}
                \nabla_{{\bf p}_1} C_1({\bf p}) &  \nabla_{{\bf p}_1} C_2({\bf p}) & \hdots  & \nabla_{{\bf p}_1} C_N({\bf p}) \\
                \nabla_{{\bf p}_2} C_1({\bf p}) &  \nabla_{{\bf p}_2} C_2({\bf p}) & \hdots  & \nabla_{{\bf p}_2} C_N({\bf p}) \\
                \vdots & \vdots & & \vdots \\
                \nabla_{{\bf p}_N} C_1({\bf p})  &  \nabla_{{\bf p}_N} C_2({\bf p}) & \hdots  & \nabla_{{\bf p}_N} C_N({\bf p}) \\
            \end{pmatrix}
            \begin{pmatrix}
                \lambda_1 \\
                \lambda_2 \\
                \vdots \\
                \lambda_N
            \end{pmatrix}
            =
            \begin{pmatrix}
                0 \\
                0 \\
                \vdots \\
                0
            \end{pmatrix}          
        \end{eqnarray*}
\end{itemize}

\underline{Kernels}
\begin{itemize}
    \item In two dimensions: given an unnormalized kernel $W'(r, h)$. $W'(r, h)$ is normalized by $2\pi\int_{r=0}^h r W'(r, h) dr$.
    \item \emph{poly6} Kernel (2D)
        \begin{eqnarray}
        W_{poly6\_2D}(r, h) = \frac{4}{\pi h^8}\left\{\begin{array}{ll} (h^2 - r^2)^3 & {\rm if\ } r < h \\ 0 & {\rm otherwise}\end{array}\right.
        \end{eqnarray}
\end{itemize}


\end{document}
